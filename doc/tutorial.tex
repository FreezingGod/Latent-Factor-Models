\documentclass[10pt]{article}

\usepackage{amsmath}
\usepackage{amssymb}
\usepackage{bm}
\newcommand{\parahead}[1]{\vspace{0.15in}\noindent{\bf #1:}}

\begin{document}
\title{How to Run the Latent Factor Model Code}
\author{Bee-Chung Chen}
\maketitle

\section{Preparation}

Before you can use this code to fit any model, you need to install R (with R version $\geq$ 2.10.1) and compile the C/C++ code in this package.

\subsection{Install R}

To install R, go to {\tt http://www.r-project.org/}.  Click CRAN on the left panel.  Pick a CRAN mirror.  Then, install R from the R source code.

Alternatively, you can install R using linux's package management software.  In this case, please install {\tt r-base}, {\tt r-base-core}, {\tt r-base-dev}, {\tt r-recommended}.

After installing R, install the following R packages: {\tt Matrix} and {\tt glmnet}.  Notice that these two packages are not required if you do not need to handle sparse feature vectors or matrices.  To install these R packages, use the following commands in R.\\
{\tt
> install.packages("Matrix");\\
> install.packages("glmnet");
}

\subsection{Compile C/C++ Code}

This is extremely simple.  Just type {\tt make} in the top-level directory (i.e., the directory that contains LICENSE, README, Makefile, Makevars, etc.).

\section{Bias-Smoothed Tensor Model}

The bias-smoothed tensor (BST) model~\cite{bst:kdd11} includes the regression-based latent factor model (RLFM)~\cite{rlfm:kdd09} and regular matrix factorization models as special cases.  In fact, the BST model presented here is a bit more general than the model presented in~\cite{bst:kdd11}.  In the following, I demonstrate how to fit such a model and its special cases.  The R code of this section can be found in {\tt src/R/examples/tutorial-BST.R}.

\subsection{Model}

We first specify the model in its most general form and then describe special cases.  Let $y_{ijkpq}$ denote the {\em response} (e.g., rating) that {\em source node} $i$ (e.g., user $i$) gives {\em destination node} $j$ (e.g., item $j$) in {\em context} $(k,p,q)$, where the context is specified by a three dimensional vector:
\begin{itemize}
\item {\em Edge context} $k$ specifies the context when the response occurs on the edge from node $i$ to node $j$; e.g., the rating on the edge from user $i$ to item $j$ was given when $i$ saw $j$ on web page $k$.
\item {\em Source context} $p$ specifies the context (or mode) of the source node $i$ when this node gives the response; e.g., $p$ represents the category of item $j$, meaning that user $i$ are in different modes when rating items in different categories.
\item {\em Destination context} $q$ specifies the context (or mode) of the destination node $j$ when this node receives the response; e.g., $q$ represents the user segment that user $i$ belongs to, meaning that the response that an item receives depends on the segment that the user belongs to.
\end{itemize}
Notice that the context $(k,p,q)$ is assumed to be given and each individual response is assumed to occur in a single context.  Also note that when modeling a problem, we may not always need all the three components in the three dimensional context vector.

Because $i$ always denotes a source node (e.g., a user), $j$ always denotes a destination node (e.g., an item) and $k$ always denotes an edge context, we slightly abuse our notation by using $\bm{x}_i$ to denote the feature vector of source node $i$, $\bm{x}_j$ to denote the feature vector of destination node $j$, $\bm{x}_k$ to denote the feature vector of edge context $k$, and $\bm{x}_{ijk}$ to denote the feature vector associated with the occasion when $i$ gives $j$ the response in context $k$.

\parahead{Response model}
For numeric response, we use the Gaussian response model; for binary response, we use the logistic response model.
\begin{equation*}
y_{ijkpq} \sim \mathcal{N}(\mu_{ijkpq},~ \sigma^2_{y}) ~\textrm{ or }~
y_{ijkpq} \sim \textit{Bernoulli}((1 + \exp(-\mu_{ijkpq}))^{-1}),
\end{equation*}
where $\mu_{ijkpq} = \bm{x}'_{ijk} \bm{b} + \alpha_{ip} + \beta_{jq} + \gamma_{k} + \left<\bm{u}_i, \bm{v}_j, \bm{w}_k\right>$.  Note that $\left<\bm{u}_i, \bm{v}_j, \bm{w}_k\right> = \sum_{\ell} \bm{u}_i[\ell]\, \bm{v}_j[\ell]\, \bm{w}_k[\ell]$ is a form of the tensor product of three vectors $\bm{u}_i$, $\bm{v}_j$ and $\bm{w}_k$, where $\bm{u}_i[\ell]$ denotes the $\ell$th element in vector $\bm{u}_i$.
For ease of exposition, we use the following notation to represent both the Gaussian and logistic models.
\begin{equation}
y_{ijkpq} \sim \bm{x}'_{ijk} \bm{b} + \alpha_{ip} + \beta_{jq} + \gamma_{k} + \left<\bm{u}_i, \bm{v}_j, \bm{w}_k\right>, \label{eq:uvw-model}
\end{equation}
where $\bm{b}$ is the regression coefficient vector on feature vector $\bm{x}_{ijk}$; $\alpha_{ip}$ is the latent factor of source node $i$ in source context $p$; $\beta_{jq}$ is the latent factor of destination node $j$ in destination context $q$; $\gamma_{k}$ is the latent factor of edge context $k$; $\bm{u}_i$, $\bm{v}_j$ and $\bm{w}_k$ are the latent factor vectors of source node $i$, destination node $j$ and edge context $k$, respectively.  Note that these latent factors and regression coefficients will be learned from data.

\parahead{Regression Priors}
The priors of the latent factors are specified in the following:
\begin{align}
\alpha_{ip} & \sim \mathcal{N}(\bm{g}_{p}^\prime \bm{x}_{i} + q_{p} \alpha_i, ~\sigma_{\alpha,p}^2),
	~~~~ \alpha_i \sim \mathcal{N}(0, 1) \label{eq:alpha} \\
\beta_{jq} & \sim \mathcal{N}(\bm{d}_{q}^\prime \bm{x}_{j} + r_{q} \beta_j, ~\sigma_{\beta,q}^2),
	~~~~ \beta_j  \sim \mathcal{N}(0, 1) \label{eq:beta} \\
\bm{u}_{i} & \sim \mathcal{N}(G' \bm{x}_i, \,\sigma_{u}^2 I), ~~~
\bm{v}_{j} \sim \mathcal{N}(D' \bm{x}_j, \,\sigma_{v}^2 I), ~~~
\bm{w}_{k} \sim \mathcal{N}(H' \bm{x}_k, \,\sigma_{w}^2 I), \label{eq:uvw}
\end{align}
where $\bm{g}_p$, $q_p$, $\bm{d}_q$, $r_q$, $G$, $D$ and $H$ are regression coefficient vectors and matrices.  These regression coefficients will be learned from data and provide the ability to make predictions for users or items that do not appear in training data.  The factors of these new users or items will be predicted based on their features through regression.

\subsection{Toy Dataset}

Before we describe how to fit a model, we first introduce a toy dataset.  You can find this toy dataset in the following directory:
\begin{verbatim}
test-data/multicontext_model/simulated-mtx-uvw-10K
\end{verbatim}
Please read the README file there to better understand this dataset, which was created by running the following R script.  Please do not rerun this R script.
\begin{verbatim}
src/unit-test/multicontext_model/create-simulated-data-1.R
\end{verbatim}
This is a simulated dataset; i.e., the response values $y_{ijkpq}$ are generated according to a ground-truth model.  To see the ground-truth, run the following commands in R.
{\small
\begin{verbatim}
> load("test-data/multicontext_model/simulated-mtx-uvw-10K/ground-truth.RData");
> str(factor);
> str(param)
\end{verbatim}
}

\bibliographystyle{abbrv}
\bibliography{references}

\end{document}
